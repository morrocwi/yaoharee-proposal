
\documentclass[12pt,a4paper]{article}
\usepackage[T1]{fontenc}
\usepackage[utf8]{inputenc}
\usepackage{lmodern,microtype}
\usepackage[a4paper,margin=1in]{geometry}
\usepackage{amsmath,amssymb,mathtools,bm,siunitx,booktabs,graphicx}
\usepackage[dvipsnames]{xcolor}
\usepackage{hyperref}
\usepackage{titlesec}
\usepackage{enumitem}
\usepackage{caption}
\hypersetup{colorlinks=true,linkcolor=MidnightBlue,citecolor=ForestGreen,urlcolor=RoyalBlue}
\sisetup{detect-all,output-decimal-marker={.}}

\title{\Large\bfseries
L4 — Phase–Amplitude Persistence in Human EEG (N=36)\\[3pt]
\large Empirical Validation of the Lyapunov–Scale–Lock Framework (Yaoharee Cosmo v3.2–MVF)}
\author{Yaoharee Lahtee\\
\textit{Cosmo @ Home (Open Science Initiative), Bangkok, Thailand}\\
ORCID: \url{https://orcid.org/0009-0005-3861-0626}\\
Email: \texttt{arayawedding@gmail.com}}
\date{November 2025}

\begin{document}
\maketitle

\begin{abstract}
\noindent
This study presents the first large-scale empirical validation (N=36) of predictions from the \emph{Yaoharee Cosmo Minimal-Viable Foundation} (L0–L3). 
It tests whether emergent phase coherence in EEG is more temporally persistent than amplitude energy. 
Nineteen-channel EEG signals (Fp1–O2, 160 Hz) were analyzed across α, β, and γ bands using Hilbert-phase decomposition.
For each region of interest (ROI), two metrics were estimated: 
phase persistence (\(\tau_\phi\)) from exponential relaxation of \(1-\rho(t)\),
and envelope persistence (\(\tau_{\mathrm{env}}\)) from autocorrelation decay.
Across all ROIs and bands, \(\tau_\phi > \tau_{\mathrm{env}}\) (p < 0.01), confirming that informational coherence persists longer than energetic amplitude — a “coherent yet volatile” regime consistent with the Lyapunov dissipation law \(\dot V \le 0\).
All data and code are openly released under CC BY 4.0.
\end{abstract}

\noindent\textbf{Keywords:} EEG, phase locking, Hilbert transform, Lyapunov dissipation, coherence, open science

\section{Methods}
\subsection{Dataset}
Data: \textit{Complete EEG Dataset} (Kaggle), from \emph{Data} 4(1):14 (2019) and PhysioNet/PhysioBank \cite{Goldberger2000}.  
N = 36 participants, 19 electrodes (Fp1–O2), sampling 160 Hz.

\subsection{Processing}
Signals were filtered into α (8–12 Hz), β (13–30 Hz), γ (30–45 Hz).  
Hilbert transform extracted analytic phase and envelope.
ROIs: Frontal, Central, Temporal, Parietal, Occipital.

\subsection{Metrics}
Phase-lock order \(\rho(t)=\left|\frac{1}{K}\sum e^{i\phi_k(t)}\right|\);  
Phase persistence \(\tau_\phi\): from \(1-\rho(t)\approx ae^{-t/\tau_\phi}+b\);  
Envelope persistence \(\tau_{\mathrm{env}}\): lag where autocorrelation(A) = \(1/e\).

\section{Results}
\begin{table}[h!]
\centering
\caption{Mean ± SD of phase (\(\tau_\phi\)) and envelope (\(\tau_{\mathrm{env}}\)) persistence across ROIs and bands (N = 36).}
\begin{tabular}{lccc}
\toprule
ROI & Alpha (8–12 Hz) & Beta (13–30 Hz) & Gamma (30–45 Hz)\\
\midrule
\input{L4_table_rows_N36.txt}
\bottomrule
\end{tabular}
\end{table}

\begin{figure}[h!]
\centering
\includegraphics[width=0.9\linewidth]{l4_eeg.png}
\caption{L4 (N = 36) Core Finding: Phase Persistence (\(\tau_\phi\)) vs. Envelope Persistence (\(\tau_{\mathrm{env}}\)).
Bars show mean ± SD. Dark = \(\tau_\phi\); light = \(\tau_{\mathrm{env}}\).}
\label{fig:l4bar}
\end{figure}

\section{Discussion}
The consistent inequality \(\tau_\phi>\tau_{\mathrm{env}}\) empirically validates the dissipative-path postulate of L0–L3.  
Phase synchrony persists longer than amplitude energy, indicating that informational order outlives energetic expression.  
Occipital α bands show strongest coherence (\(\rho\approx0.73, \tau_\phi\approx0.42\) s) while frontal γ activity is most transient (\(\tau_\phi\approx0.15\) s).  
This supports the Lyapunov functional \(V=\frac12(A\varepsilon-\delta)^{\!\top}W(A\varepsilon-\delta)\) with \(\dot V\le0\), where energy decays but relational structure persists.

\textbf{Interpretation.}  
Phase persistence (\(\tau_\phi\)) represents stability of informational coherence; envelope persistence (\(\tau_{\mathrm{env}}\)) represents energy dissipation.  
Their separation quantitatively supports the principle “Emergence ≠ Stability.”  
Hence, coherence is not a static equilibrium but a self-restoring dynamic state.

\textbf{Outlook.}  
Future work will generalize this dual-metric framework to other coherence domains—laser fields, Schumann resonances, gravitational signals—to test universality of \(\tau_\phi>\tau_{\mathrm{env}}\) across physical and informational scales.

\section*{Open Science Statement}
\textit{Our Methodology—Citizen Open Science Work @ Home with AI:}
We encourage transparent, reproducible exploration from home using AI under FAIR principles (CC BY 4.0).  
Repository: \url{https://github.com/morrocwi/yaoharee-proposal}.

\begin{thebibliography}{9}
\bibitem{Goldberger2000}
Goldberger A. L. et al. (2000). PhysioBank, PhysioToolkit, and PhysioNet: Components of a new research resource for complex physiologic signals. \textit{Circulation}, 101(23), e215–e220.

\bibitem{MDPI2019}
Rahman A., Arif S. M., \& Ahmed F. (2019). A Comprehensive Survey on EEG Data Preprocessing and Transformation Methods. \textit{Data}, 4(1), 14.

\bibitem{Yaoharee2025}
Lahtee Y. (2025). \textit{L0–L3: Minimal-Viable Foundation for the Emergent Coherence Field.} Zenodo. DOI: 10.5281/zenodo.17495915.
\end{thebibliography}

\end{document}
